%% BioMed_Central_Tex_Template_v1.06
%%                                      %
%  bmc_article.tex            ver: 1.06 %
%                                       %

%%IMPORTANT: do not delete the first line of this template
%%It must be present to enable the BMC Submission system to
%%recognise this template!!

%%%%%%%%%%%%%%%%%%%%%%%%%%%%%%%%%%%%%%%%%
%%                                     %%
%%  LaTeX template for BioMed Central  %%
%%     journal article submissions     %%
%%                                     %%
%%          <8 June 2012>              %%
%%                                     %%
%%                                     %%
%%%%%%%%%%%%%%%%%%%%%%%%%%%%%%%%%%%%%%%%%

%%%%%%%%%%%%%%%%%%%%%%%%%%%%%%%%%%%%%%%%%%%%%%%%%%%%%%%%%%%%%%%%%%%%%
%%                                                                 %%
%% For instructions on how to fill out this Tex template           %%
%% document please refer to Readme.html and the instructions for   %%
%% authors page on the biomed central website                      %%
%% https://www.biomedcentral.com/getpublished                      %%
%%                                                                 %%
%% Please do not use \input{...} to include other tex files.       %%
%% Submit your LaTeX manuscript as one .tex document.              %%
%%                                                                 %%
%% All additional figures and files should be attached             %%
%% separately and not embedded in the \TeX\ document itself.       %%
%%                                                                 %%
%% BioMed Central currently use the MikTex distribution of         %%
%% TeX for Windows) of TeX and LaTeX.  This is available from      %%
%% https://miktex.org/                                             %%
%%                                                                 %%
%%%%%%%%%%%%%%%%%%%%%%%%%%%%%%%%%%%%%%%%%%%%%%%%%%%%%%%%%%%%%%%%%%%%%

%%% additional documentclass options:
%  [doublespacing]
%  [linenumbers]   - put the line numbers on margins

%%% loading packages, author definitions

%\documentclass[twocolumn]{bmcart}% uncomment this for twocolumn layout and comment line below
\documentclass{bmcart}

%%% Load packages
\usepackage{array}
\usepackage{amsthm,amsmath}
%\RequirePackage[numbers]{natbib}
%\RequirePackage[authoryear]{natbib}% uncomment this for author-year bibliography
%\RequirePackage{hyperref}
\usepackage[utf8]{inputenc} %unicode support
%\usepackage[applemac]{inputenc} %applemac support if unicode package fails
%\usepackage[latin1]{inputenc} %UNIX support if unicode package fails

%%%%%%%%%%%%%%%%%%%%%%%%%%%%%%%%%%%%%%%%%%%%%%%%%
%%                                             %%
%%  If you wish to display your graphics for   %%
%%  your own use using includegraphic or       %%
%%  includegraphics, then comment out the      %%
%%  following two lines of code.               %%
%%  NB: These line *must* be included when     %%
%%  submitting to BMC.                         %%
%%  All figure files must be submitted as      %%
%%  separate graphics through the BMC          %%
%%  submission process, not included in the    %%
%%  submitted article.                         %%
%%                                             %%
%%%%%%%%%%%%%%%%%%%%%%%%%%%%%%%%%%%%%%%%%%%%%%%%%

\def\includegraphic{}
\def\includegraphics{}

%%% Put your definitions there:
\startlocaldefs
\endlocaldefs

%%% Begin ...
\begin{document}

%%% Start of article front matter
\begin{frontmatter}

\begin{fmbox}
\dochead{Research}

%%%%%%%%%%%%%%%%%%%%%%%%%%%%%%%%%%%%%%%%%%%%%%
%%                                          %%
%% Enter the title of your article here     %%
%%                                          %%
%%%%%%%%%%%%%%%%%%%%%%%%%%%%%%%%%%%%%%%%%%%%%%

\title{Evaluating trait-based sets for taxonomic enrichment analysis applied to human microbiome data sets}

%%%%%%%%%%%%%%%%%%%%%%%%%%%%%%%%%%%%%%%%%%%%%%
%%                                          %%
%% Enter the authors here                   %%
%%                                          %%
%% Specify information, if available,       %%
%% in the form:                             %%
%%   <key>={<id1>,<id2>}                    %%
%%   <key>=                                 %%
%% Comment or delete the keys which are     %%
%% not used. Repeat \author command as much %%
%% as required.                             %%
%%                                          %%
%%%%%%%%%%%%%%%%%%%%%%%%%%%%%%%%%%%%%%%%%%%%%%

\author[
  addressref={aff1,aff2},                   
  email={quang.p.nguyen.gr@dartmouth.edu}   % email address
]{\inits{Q.P.N.}\fnm{Quang P.} \snm{Nguyen}}
\author[
  addressref={aff1,aff2},
  email={anne.g.hoen@dartmouth.edu}
]{\inits{A.G.H.}\fnm{Anne G.} \snm{Hoen}}
\author[
  addressref={aff1},
  corref={aff1},
  email={hildreth.r.frost@dartmouth.edu}
]{\inits{H.R.F.}\fnm{H. Robert} \snm{Frost}}

%%%%%%%%%%%%%%%%%%%%%%%%%%%%%%%%%%%%%%%%%%%%%%
%%                                          %%
%% Enter the authors' addresses here        %%
%%                                          %%
%% Repeat \address commands as much as      %%
%% required.                                %%
%%                                          %%
%%%%%%%%%%%%%%%%%%%%%%%%%%%%%%%%%%%%%%%%%%%%%%

\address[id=aff1]{%                           % unique id
  \orgdiv{Department of Biomedical Data Science},             % department, if any
  \orgname{Dartmouth College},          % university, etc
  \city{Hanover},                              % city
  \cny{NH, USA}                                    % country
}
\address[id=aff2]{%
  \orgdiv{Department of Epidemiology},
  \orgname{Dartmouth College},
  %\street{},
  %\postcode{}
  \city{Hanover},
  \cny{NH, USA}
}

%%%%%%%%%%%%%%%%%%%%%%%%%%%%%%%%%%%%%%%%%%%%%%
%%                                          %%
%% Enter short notes here                   %%
%%                                          %%
%% Short notes will be after addresses      %%
%% on first page.                           %%
%%                                          %%
%%%%%%%%%%%%%%%%%%%%%%%%%%%%%%%%%%%%%%%%%%%%%%

%\begin{artnotes}
%%\note{Sample of title note}     % note to the article
%\note[id=n1]{Equal contributor} % note, connected to author
%\end{artnotes}

\end{fmbox}% comment this for two column layout

%%%%%%%%%%%%%%%%%%%%%%%%%%%%%%%%%%%%%%%%%%%%%%%
%%                                           %%
%% The Abstract begins here                  %%
%%                                           %%
%% Please refer to the Instructions for      %%
%% authors on https://www.biomedcentral.com/ %%
%% and include the section headings          %%
%% accordingly for your article type.        %%
%%                                           %%
%%%%%%%%%%%%%%%%%%%%%%%%%%%%%%%%%%%%%%%%%%%%%%%

\begin{abstractbox}

\begin{abstract} % abstract
\parttitle{Background} Set-based pathway analysis is a powerful tool that allows researchers to summarize complex genomic variables in the form of biologically interpretable sets. Since the microbiome is characterized by a high degree of inter-individual variability in taxonomic compositions, applying enrichment methods using on functionally driven taxon sets can increase both the reproducibility and interpretability of microbiome association studies. However, there is still an open question of which knowledge base to utilize for set construction. Here, we evaluate microbial trait databases, which aggregates experimentally determined microbial phenotypes, as a potential avenue for meaningful construction of taxon sets.  

\parttitle{Methods} Using publicly available microbiome sequencing data sets (both 16S rRNA gene metabarcoding and whole-genome metagenomics), we assessed these trait-based sets on three aspects: first, do they cover the diversity of microbes obtained from a typical data set; second, do they confer additional predictive power on disease prediction tasks when assessed against measured pathway abundances and PICRUSt2 prediction; third, for sets that are found to be enriched, are pathways corresponding to the trait also have increased abundances.     

\parttitle{Results}  
\parttitle{Conclusions}

\end{abstract}

%%%%%%%%%%%%%%%%%%%%%%%%%%%%%%%%%%%%%%%%%%%%%%
%%                                          %%
%% The keywords begin here                  %%
%%                                          %%
%% Put each keyword in separate \kwd{}.     %%
%%                                          %%
%%%%%%%%%%%%%%%%%%%%%%%%%%%%%%%%%%%%%%%%%%%%%%

\begin{keyword}
\kwd{microbiome}
\kwd{enrichment analysis}
\kwd{trait database}
\end{keyword}

% MSC classifications codes, if any
%\begin{keyword}[class=AMS]
%\kwd[Primary ]{}
%\kwd{}
%\kwd[; secondary ]{}
%\end{keyword}

\end{abstractbox}
%
%\end{fmbox}% uncomment this for two column layout

\end{frontmatter}

%%%%%%%%%%%%%%%%%%%%%%%%%%%%%%%%%%%%%%%%%%%%%%%%
%%                                            %%
%% The Main Body begins here                  %%
%%                                            %%
%% Please refer to the instructions for       %%
%% authors on:                                %%
%% https://www.biomedcentral.com/getpublished %%
%% and include the section headings           %%
%% accordingly for your article type.         %%
%%                                            %%
%% See the Results and Discussion section     %%
%% for details on how to create sub-sections  %%
%%                                            %%
%% use \cite{...} to cite references          %%
%%  \cite{koon} and                           %%
%%  \cite{oreg,khar,zvai,xjon,schn,pond}      %%
%%                                            %%
%%%%%%%%%%%%%%%%%%%%%%%%%%%%%%%%%%%%%%%%%%%%%%%%

%%%%%%%%%%%%%%%%%%%%%%%%% start of article main body
% <put your article body there>

%%%%%%%%%%%%%%%%
%% Background %%
%%
\section*{Introduction}

\section*{Material and methods}
\subsection*{Trait database}  

We utilized pre-compiled trait databases from previous publications, Madin et al. \cite{madin2020synthesis} and Weissman et al \cite{weissman2021exploring}. Madin et al. was chosen since it is the most  comprehensive compilation of microbial (bacteria and archaea) physiological traits based on existing sources to date. On the other hand, Weissman et al. provides a novel human microbiome specific curation based on the latest version of Bergey's manual. We combined both databases into a joint knowledge base and constructed sets for each available categorical trait. Additionally for the Madin et al. database, we updated data entries sourced from GOLD \cite{mukherjee2021genomes} due to the fact that compared to other compiled sources, GOLD is continuously updated via community submissions.  

\noindent All traits are defined at the species level via NCBI identifiers. However, due to restrictions for 16S rRNA gene sequencing data sets to resolve beyond the genus level \cite{johnson2019evaluation}, we also assigned traits to each genera based on a two-step process: 
\begin{itemize}
    \item A hypergeometric test to ascertain whether the genera is underrepresented in the database based on the total number of species assigned to that genera in NCBI Taxonomy \cite{schoch2020ncbi} compared to our trait database. If a genera is underrepresented in our database, the trait is not assigned to the genera. Specifically we assessed $P(X \leq x)$ at $\alpha = 0.05$ where $X \sim Hypergeometric(k, N, K)$, with $x$ as the total number of species assigned to that genera in the database, $k$ as the total number of species in the database, $N$ as the total number of species in NCBI Taxonomy, and $K$ as the total number of species assigned to the genera in NCBI Taxonomy.
    \item For all genera that are well represented in the database, we then assessed the proportion of species under that genera has the trait. If over 95\% of species have the trait, then the trait is assigned to the genera. 
\end{itemize}

\subsection*{Data sets}  
We evaluated trait-based sets on publicly available 16S rRNA gene metabarcoding and whole-genome metagenomic data sets. 

\noindent To assess trait annotation coverage, we utilized data from the Human Microbiome Project (HMP) \cite{


\begin{itemize}
    \item 
\end{itemize}

\subsection*{Comparison approaches}  

\subsection*{Coverage analysis}  

\subsection*{Prediction analysis}  

\subsection*{Concordance analysis}

\section*{Results}

\section*{Discussion}

\section*{Appendix}
Text for this section\ldots

%%%%%%%%%%%%%%%%%%%%%%%%%%%%%%%%%%%%%%%%%%%%%%
%%                                          %%
%% Backmatter begins here                   %%
%%                                          %%
%%%%%%%%%%%%%%%%%%%%%%%%%%%%%%%%%%%%%%%%%%%%%%

\begin{backmatter}

\section*{Acknowledgements}%% if any
Text for this section\ldots

\section*{Funding}%% if any
Text for this section\ldots

\section*{Abbreviations}%% if any
Text for this section\ldots

\section*{Availability of data and materials}%% if any
Text for this section\ldots

\section*{Ethics approval and consent to participate}%% if any
Text for this section\ldots

\section*{Competing interests}
The authors declare that they have no competing interests.

\section*{Consent for publication}%% if any
Text for this section\ldots

\section*{Authors' contributions}
Text for this section \ldots

\section*{Authors' information}%% if any
Text for this section\ldots

%%%%%%%%%%%%%%%%%%%%%%%%%%%%%%%%%%%%%%%%%%%%%%%%%%%%%%%%%%%%%
%%                  The Bibliography                       %%
%%                                                         %%
%%  Bmc_mathpys.bst  will be used to                       %%
%%  create a .BBL file for submission.                     %%
%%  After submission of the .TEX file,                     %%
%%  you will be prompted to submit your .BBL file.         %%
%%                                                         %%
%%                                                         %%
%%  Note that the displayed Bibliography will not          %%
%%  necessarily be rendered by Latex exactly as specified  %%
%%  in the online Instructions for Authors.                %%
%%                                                         %%
%%%%%%%%%%%%%%%%%%%%%%%%%%%%%%%%%%%%%%%%%%%%%%%%%%%%%%%%%%%%%

% if your bibliography is in bibtex format, use those commands:
\bibliographystyle{bmc-mathphys} % Style BST file (bmc-mathphys, vancouver, spbasic).
\bibliography{references}      % Bibliography file (usually '*.bib' )
% for author-year bibliography (bmc-mathphys or spbasic)
% a) write to bib file (bmc-mathphys only)
% @settings{label, options="nameyear"}
% b) uncomment next line
%\nocite{label}

% or include bibliography directly:
% \begin{thebibliography}
% \bibitem{b1}
% \end{thebibliography}

%%%%%%%%%%%%%%%%%%%%%%%%%%%%%%%%%%%
%%                               %%
%% Figures                       %%
%%                               %%
%% NB: this is for captions and  %%
%% Titles. All graphics must be  %%
%% submitted separately and NOT  %%
%% included in the Tex document  %%
%%                               %%
%%%%%%%%%%%%%%%%%%%%%%%%%%%%%%%%%%%

%%
%% Do not use \listoffigures as most will included as separate files

\section*{Figures}
  \begin{figure}[h!]
  \caption{Sample figure title}
\end{figure}

\begin{figure}[h!]
  \caption{Sample figure title}
\end{figure}

%%%%%%%%%%%%%%%%%%%%%%%%%%%%%%%%%%%
%%                               %%
%% Tables                        %%
%%                               %%
%%%%%%%%%%%%%%%%%%%%%%%%%%%%%%%%%%%

%% Use of \listoftables is discouraged.
%%
\section*{Tables}
\begin{table}[h!]
\caption{Sample table title. This is where the description of the table should go}
  \begin{tabular}{cccc}
    \hline
    & B1  &B2   & B3\\ \hline
    A1 & 0.1 & 0.2 & 0.3\\
    A2 & ... & ..  & .\\
    A3 & ..  & .   & .\\ \hline
  \end{tabular}
\end{table}

%%%%%%%%%%%%%%%%%%%%%%%%%%%%%%%%%%%
%%                               %%
%% Additional Files              %%
%%                               %%
%%%%%%%%%%%%%%%%%%%%%%%%%%%%%%%%%%%

\section*{Additional Files}
  \subsection*{Additional file 1 --- Sample additional file title}
    Additional file descriptions text (including details of how to
    view the file, if it is in a non-standard format or the file extension).  This might
    refer to a multi-page table or a figure.

  \subsection*{Additional file 2 --- Sample additional file title}
    Additional file descriptions text.

\end{backmatter}
\end{document}
