\documentclass[10pt,letterpaper]{article}
\usepackage[top=0.85in,left=2.75in,footskip=0.75in,marginparwidth=2in]{geometry}

% use Unicode characters - try changing the option if you run into troubles with special characters (e.g. umlauts)
\usepackage[utf8]{inputenc}

% clean citations
\usepackage{cite}

% hyperref makes references clicky. use \url{www.example.com} or \href{www.example.com}{description} to add a clicky url
\usepackage{nameref,hyperref}

% line numbers
\usepackage[right]{lineno}

% improves typesetting in LaTeX
\usepackage{microtype}
\DisableLigatures[f]{encoding = *, family = * }

% text layout - change as needed
\raggedright
\setlength{\parindent}{0.5cm}
\textwidth 5.25in 
\textheight 8.75in

% Remove % for double line spacing
%\usepackage{setspace} 
%\doublespacing

% use adjustwidth environment to exceed text width (see examples in text)
\usepackage{changepage}

% adjust caption style
\usepackage[aboveskip=1pt,labelfont=bf,labelsep=period,singlelinecheck=off]{caption}

% remove brackets from references
\makeatletter
\renewcommand{\@biblabel}[1]{\quad#1.}
\makeatother

% headrule, footrule and page numbers
\usepackage{lastpage,fancyhdr,graphicx}
\usepackage{epstopdf}
\pagestyle{myheadings}
\pagestyle{fancy}
\fancyhf{}
\rfoot{\thepage/\pageref{LastPage}}
\renewcommand{\footrule}{\hrule height 2pt \vspace{2mm}}
\fancyheadoffset[L]{2.25in}
\fancyfootoffset[L]{2.25in}

% use \textcolor{color}{text} for colored text (e.g. highlight to-do areas)
\usepackage{color}

% define custom colors (this one is for figure captions)
\definecolor{Gray}{gray}{.25}

% this is required to include graphics
\usepackage{graphicx}

% use if you want to put caption to the side of the figure - see example in text
\usepackage{sidecap}

% use for have text wrap around figures
\usepackage{wrapfig}
\usepackage[pscoord]{eso-pic}
\usepackage[fulladjust]{marginnote}
\reversemarginpar

% document begins here
\begin{document}
\vspace*{0.35in}

% title goes here:
\begin{flushleft}
{\Large
\textbf\newline{Evaluating trait-based taxon-sets for function-driven enrichment analysis for microbiome relative abundance data}
}
\newline
% authors go here:
\\
Quang Nguyen\textsuperscript{1,2},
Anne G. Hoen\textsuperscript{1,2,+},
H. Robert Frost\textsuperscript{2,+,*}
\\
\bigskip
\bf{1} Department of Epidemiology, Geisel School of Medicine at Dartmouth College
\\
\bf{2} Department of Biomedical Data Science, Geisel School of Medicine at Dartmouth College
\\
\bigskip
+ These authors jointly supervised this research \\

* hildreth.r.frost@dartmouth.edu

\end{flushleft}

\section*{Abstract}
Fitting multiple figures into very tight manuscripts while keeping it pleasant to read is challenging. Therefore figures are often simply attached to the very end of a manuscript file. While easier for the authors, this practice is inconvenient for readers. This \LaTeX template shows how to generate a compiled PDF with figures embedded into the text. It provides several examples of how to embed figures or tables directly into the text thus giving you a range of options from which you should choose the one best suited for your manuscript.

% now start line numbers
\linenumbers

% the * after section prevents numbering
\section*{Introduction}
\section*{Materials and Methods}
\subsection*{Preparing taxon sets based on trait databases}
\subsection*{Data retrieval and processing}
\subsection*{Enrichment analysis with cILR}
\subsection*{Metagenomic prediction with PiCRUST2}
\section*{Results}
\section*{Discussion}
\section*{Supporting Information}

%These commands reset the figure counter and add "S" to the figure caption (e.g. "Figure S1"). This is in case you want to add actual figures and not just captions.
\setcounter{figure}{0}
\renewcommand{\thefigure}{S\arabic{figure}}

% You can use the \nameref{label} command to cite supporting items in the text.

%\clearpage

\section*{Acknowledgments}
We thank just about everybody.

\nolinenumbers

%This is where your bibliography is generated. Make sure that your .bib file is actually called library.bib
\bibliography{microbe_trait}

%This defines the bibliographies style. Search online for a list of available styles.
\bibliographystyle{abbrv}

\end{document}

